% Options for packages loaded elsewhere
\PassOptionsToPackage{unicode}{hyperref}
\PassOptionsToPackage{hyphens}{url}
%
\documentclass[
]{book}
\usepackage{lmodern}
\usepackage{amsmath}
\usepackage{ifxetex,ifluatex}
\ifnum 0\ifxetex 1\fi\ifluatex 1\fi=0 % if pdftex
  \usepackage[T1]{fontenc}
  \usepackage[utf8]{inputenc}
  \usepackage{textcomp} % provide euro and other symbols
  \usepackage{amssymb}
\else % if luatex or xetex
  \usepackage{unicode-math}
  \defaultfontfeatures{Scale=MatchLowercase}
  \defaultfontfeatures[\rmfamily]{Ligatures=TeX,Scale=1}
\fi
% Use upquote if available, for straight quotes in verbatim environments
\IfFileExists{upquote.sty}{\usepackage{upquote}}{}
\IfFileExists{microtype.sty}{% use microtype if available
  \usepackage[]{microtype}
  \UseMicrotypeSet[protrusion]{basicmath} % disable protrusion for tt fonts
}{}
\makeatletter
\@ifundefined{KOMAClassName}{% if non-KOMA class
  \IfFileExists{parskip.sty}{%
    \usepackage{parskip}
  }{% else
    \setlength{\parindent}{0pt}
    \setlength{\parskip}{6pt plus 2pt minus 1pt}}
}{% if KOMA class
  \KOMAoptions{parskip=half}}
\makeatother
\usepackage{xcolor}
\IfFileExists{xurl.sty}{\usepackage{xurl}}{} % add URL line breaks if available
\IfFileExists{bookmark.sty}{\usepackage{bookmark}}{\usepackage{hyperref}}
\hypersetup{
  pdftitle={SOC 4650 \& 5650: Introduction to GIS},
  pdfauthor={Christopher Prener, Ph.D.},
  hidelinks,
  pdfcreator={LaTeX via pandoc}}
\urlstyle{same} % disable monospaced font for URLs
\usepackage{longtable,booktabs}
\usepackage{calc} % for calculating minipage widths
% Correct order of tables after \paragraph or \subparagraph
\usepackage{etoolbox}
\makeatletter
\patchcmd\longtable{\par}{\if@noskipsec\mbox{}\fi\par}{}{}
\makeatother
% Allow footnotes in longtable head/foot
\IfFileExists{footnotehyper.sty}{\usepackage{footnotehyper}}{\usepackage{footnote}}
\makesavenoteenv{longtable}
\usepackage{graphicx}
\makeatletter
\def\maxwidth{\ifdim\Gin@nat@width>\linewidth\linewidth\else\Gin@nat@width\fi}
\def\maxheight{\ifdim\Gin@nat@height>\textheight\textheight\else\Gin@nat@height\fi}
\makeatother
% Scale images if necessary, so that they will not overflow the page
% margins by default, and it is still possible to overwrite the defaults
% using explicit options in \includegraphics[width, height, ...]{}
\setkeys{Gin}{width=\maxwidth,height=\maxheight,keepaspectratio}
% Set default figure placement to htbp
\makeatletter
\def\fps@figure{htbp}
\makeatother
\setlength{\emergencystretch}{3em} % prevent overfull lines
\providecommand{\tightlist}{%
  \setlength{\itemsep}{0pt}\setlength{\parskip}{0pt}}
\setcounter{secnumdepth}{5}
\usepackage{booktabs}
\usepackage{amsthm}
\makeatletter
\def\thm@space@setup{%
  \thm@preskip=8pt plus 2pt minus 4pt
  \thm@postskip=\thm@preskip
}
\makeatother

% work around for errors related to the undefined shaded* enviornment:
\usepackage{color}
\usepackage{fancyvrb}
\newcommand{\VerbBar}{|}
\newcommand{\VERB}{\Verb[commandchars=\\\{\}]}
\DefineVerbatimEnvironment{Highlighting}{Verbatim}{commandchars=\\\{\}}
% Add ',fontsize=\small' for more characters per line
\usepackage{framed}
\definecolor{shadecolor}{RGB}{248,248,248}
\newenvironment{Shaded}{\begin{snugshade}}{\end{snugshade}}
\newcommand{\KeywordTok}[1]{\textcolor[rgb]{0.13,0.29,0.53}{\textbf{#1}}}
\newcommand{\DataTypeTok}[1]{\textcolor[rgb]{0.13,0.29,0.53}{#1}}
\newcommand{\DecValTok}[1]{\textcolor[rgb]{0.00,0.00,0.81}{#1}}
\newcommand{\BaseNTok}[1]{\textcolor[rgb]{0.00,0.00,0.81}{#1}}
\newcommand{\FloatTok}[1]{\textcolor[rgb]{0.00,0.00,0.81}{#1}}
\newcommand{\ConstantTok}[1]{\textcolor[rgb]{0.00,0.00,0.00}{#1}}
\newcommand{\CharTok}[1]{\textcolor[rgb]{0.31,0.60,0.02}{#1}}
\newcommand{\SpecialCharTok}[1]{\textcolor[rgb]{0.00,0.00,0.00}{#1}}
\newcommand{\StringTok}[1]{\textcolor[rgb]{0.31,0.60,0.02}{#1}}
\newcommand{\VerbatimStringTok}[1]{\textcolor[rgb]{0.31,0.60,0.02}{#1}}
\newcommand{\SpecialStringTok}[1]{\textcolor[rgb]{0.31,0.60,0.02}{#1}}
\newcommand{\ImportTok}[1]{#1}
\newcommand{\CommentTok}[1]{\textcolor[rgb]{0.56,0.35,0.01}{\textit{#1}}}
\newcommand{\DocumentationTok}[1]{\textcolor[rgb]{0.56,0.35,0.01}{\textbf{\textit{#1}}}}
\newcommand{\AnnotationTok}[1]{\textcolor[rgb]{0.56,0.35,0.01}{\textbf{\textit{#1}}}}
\newcommand{\CommentVarTok}[1]{\textcolor[rgb]{0.56,0.35,0.01}{\textbf{\textit{#1}}}}
\newcommand{\OtherTok}[1]{\textcolor[rgb]{0.56,0.35,0.01}{#1}}
\newcommand{\FunctionTok}[1]{\textcolor[rgb]{0.00,0.00,0.00}{#1}}
\newcommand{\VariableTok}[1]{\textcolor[rgb]{0.00,0.00,0.00}{#1}}
\newcommand{\ControlFlowTok}[1]{\textcolor[rgb]{0.13,0.29,0.53}{\textbf{#1}}}
\newcommand{\OperatorTok}[1]{\textcolor[rgb]{0.81,0.36,0.00}{\textbf{#1}}}
\newcommand{\BuiltInTok}[1]{#1}
\newcommand{\ExtensionTok}[1]{#1}
\newcommand{\PreprocessorTok}[1]{\textcolor[rgb]{0.56,0.35,0.01}{\textit{#1}}}
\newcommand{\AttributeTok}[1]{\textcolor[rgb]{0.77,0.63,0.00}{#1}}
\newcommand{\RegionMarkerTok}[1]{#1}
\newcommand{\InformationTok}[1]{\textcolor[rgb]{0.56,0.35,0.01}{\textbf{\textit{#1}}}}
\newcommand{\WarningTok}[1]{\textcolor[rgb]{0.56,0.35,0.01}{\textbf{\textit{#1}}}}
\newcommand{\AlertTok}[1]{\textcolor[rgb]{0.94,0.16,0.16}{#1}}
\newcommand{\ErrorTok}[1]{\textcolor[rgb]{0.64,0.00,0.00}{\textbf{#1}}}
\newcommand{\NormalTok}[1]{#1}

% create callout boxes:
\newenvironment{rmdblock}[1]
  {\begin{shaded*}
  \begin{itemize}
  \renewcommand{\labelitemi}{
    \raisebox{-.7\height}[0pt][0pt]{
      {\setkeys{Gin}{width=3em,keepaspectratio}\includegraphics{images/#1}}
    }
  }
  \item
  }
  {
  \end{itemize}
  \end{shaded*}
  }
\newenvironment{rmdnote}
  {\begin{rmdblock}{note}}
  {\end{rmdblock}}
\newenvironment{rmdtip}
  {\begin{rmdblock}{tip}}
  {\end{rmdblock}}
\newenvironment{rmdwarning}
  {\begin{rmdblock}{warning}}
  {\end{rmdblock}}

% set part and section names:
\usepackage{fancyhdr}
\renewcommand{\chaptername}{Section}
\renewcommand\thesection{\Alph{section}}
\ifluatex
  \usepackage{selnolig}  % disable illegal ligatures
\fi
\usepackage[]{natbib}
\bibliographystyle{apalike}

\title{SOC 4650 \& 5650: Introduction to GIS}
\author{Christopher Prener, Ph.D.}
\date{2021-02-08}

\begin{document}
\maketitle

\begin{center}
{\huge Preface and Warning} \\
\end{center}
\vspace{5mm}
This is the hardcopy version of the \textbf{Spring 2020} syllabus.
\vspace{5mm}
\par \noindent This \texttt{.pdf} version of the course syllabus is automatically created as part of the document generation process. It is meant for students who wish to keep a hardcopy of the course policies and planned course schedule. This may be particularly useful for honors and M.A. students who plan to continue their graduate education after SLU and hope to petition out of a basic course requirement. \textbf{Since it is automatically created, it is not optimized for easy use} - readers may notice formatting inconsitencies and stray characters that are a result of the markdown to \LaTeX{} conversion process. The web version (located at \href{https://slu-soc5650.github/syllabus/}{https://slu-soc5650.github/syllabus/}) is meant to be the version of the syllabus used for everyday reference during the semester. As such, this \texttt{.pdf} version will not be updated as the semester progresses should any changes to the course schedule be necessary.

\hypertarget{basics}{%
\chapter*{Basics}\label{basics}}
\addcontentsline{toc}{chapter}{Basics}

\hypertarget{course-meetings}{%
\subsection*{Course Meetings}\label{course-meetings}}
\addcontentsline{toc}{subsection}{Course Meetings}

\emph{When:} Mondays, 4:15pm CST to 7:00pm CST

\emph{Where:} Zoom (see Blackboard for the link and password)

\hypertarget{course-website}{%
\subsection*{Course Website}\label{course-website}}
\addcontentsline{toc}{subsection}{Course Website}

\url{https://slu-soc5650.github.io}

\hypertarget{chriss-information}{%
\subsection*{Chris's Information}\label{chriss-information}}
\addcontentsline{toc}{subsection}{Chris's Information}

\begin{rmdwarning}
Please note that I am not on-campus this semester due to the continuing
COVID-19 pandemic. All course meetings, office hours, and individual
student meetings will occur virtually via Zoom.
\end{rmdwarning}

\emph{Office:} 1918 Morrissey Hall

\emph{Email:} \href{mailto:chris.prener@slu.edu}{\nolinkurl{chris.prener@slu.edu}}

\emph{GitHub:} \texttt{@chris-prener}

\textbf{Office Hours, Appointment Only:} Wednesdays, 9:00 AM CST to 10:00 AM CST; sign-up via Calendly to receive personalized calendar and Zoom invitations (SLU log-in required)

\textbf{Office Hours, Drop-in:} Wednesdays, 10:00 AM CST to 10:45 AM CST via Zoom (SLU log-in required; you will be admitted to a waiting room)

\hypertarget{hardcopy-syllabus}{%
\section*{Hardcopy Syllabus}\label{hardcopy-syllabus}}
\addcontentsline{toc}{section}{Hardcopy Syllabus}

If you would like to keep a record of the syllabus, there is a \texttt{.pdf} download button () in the top toolboar. This may be particularly useful for honors and M.A.~students who plan to continue their graduate education after SLU and hope to petition out of a basic statistics requirement. This document will contain a ``snapshot'' of the course policies and planned schedule as of the beginning of the semester but will not be subsequently updated. See the ``Preface and Warning'' on page 2 of the \texttt{.pdf} for additional details.

\hypertarget{change-log}{%
\section*{Change Log}\label{change-log}}
\addcontentsline{toc}{section}{Change Log}

\begin{itemize}
\tightlist
\item
  January 27, 2021 - Update for Spring 2021 semester
\end{itemize}

\hypertarget{license}{%
\section*{License}\label{license}}
\addcontentsline{toc}{section}{License}

Copyright © 2016-2021 \href{https://chris-prener.github.io}{Christopher G. Prener}

This work is licensed under a Creative Commons Attribution-ShareAlike 4.0 International License.

\hypertarget{part-syllabus}{%
\part{Syllabus}\label{part-syllabus}}

\hypertarget{course-introduction}{%
\chapter{Course Introduction}\label{course-introduction}}

\begin{quote}
{[}One{]} cannot understand social life without understanding the arrangements of particular social actors in particular social times and places\ldots{}\textbf{\emph{social facts are located}}.
\end{quote}

\textbf{Andrew Abbot (1997)}

This class introduces both the theoretical and technical skills that constitute the growing field of Geographic Information Science (GISc). Techniques introduced include data cleaning and management, map production and cartography, and the manipulation of both tabular and spatial data. The course incorporates a wide variety of social, economic, health, urban, meteorological, and environmental data. These data are mapped at a variety of extents, from the City of St.~Louis to the St.~Louis Metropolitan region, Missouri, all United States counties, and all U.S. states.

\hypertarget{two-courses-one-goal}{%
\section{Two Courses, One Goal}\label{two-courses-one-goal}}

Students will quickly notice that this course has two numbers. SOC 4650 is the undergraduate section, and SOC 5650 is the graduate section. This quickly leads to anxiety for some students, who worry they have signed up for the wrong class (occasionally this is not misplaced anxiety - make sure you are enrolled in the correct section!) or who worry that they are taking a class that is not appropriate for their skill level. This class is designed for students with little to no background in GIS and scientific computing more generally. For those students, the level is largely irrelevant - undergraduate and graduate students who have not been exposed to these ideas need to cover the same material.

Graduate students who take this class will have to do some additional work - the final project is more rigorous than the project that undergraduates will complete. Otherwise, the course is the same because what content students need is largely the same as well.

\hypertarget{course-objectives}{%
\section{Course Objectives}\label{course-objectives}}

This course has five intertwined objectives. After completing the course, students will be able to:

\begin{enumerate}
\def\labelenumi{\arabic{enumi}.}
\item
  \emph{Geographic information science}: Describe the concepts that form the foundation of GISc work.
\item
  \emph{Data management}: Perform basic data cleaning and geoprocessing tasks using \texttt{R}.
\item
  \emph{Data visualization}: Create and present visualizations of spatial data using \texttt{R} and ArcGIS Online.
\item
  \emph{Analysis development}: Apply techniques that make GISc work more reproducible, accurate, and collaborative using GitHub, \texttt{R}, Markdown, and other tools.
\item
  \emph{Research synthesis}: Plan and implement a spatial data analysis project that utilizes the techniques described throughout the course.
\end{enumerate}

\hypertarget{core-resources}{%
\section{Core Resources}\label{core-resources}}

There are two core documents and resources for this course. This \textbf{Syllabus} sets out core expectations and policies for the course - i.e.~what is \emph{required} for this course. It includes a \textbf{Reading List} that contains topics, readings (both required and optional), and assignment due dates for each week. Once the semester starts, these documents will only be updated if a schedule change is necessary.

In addition to these documents, regular updates will be provided on the \href{https://slu-soc5650.github.io}{\textbf{course website}}. Each class meeting will have a corresponding page on the site that includes links to handouts, YouTube videos, sample code, and additional descriptions of concepts covered in class. If bugs or issues arise, they will be documented along with solutions here as well. Please check the website regularly for updates and new content.

\hypertarget{readings}{%
\section{Readings}\label{readings}}

There are one books required for this course with an optional second book. Each book has been selected to correspond with one or more of the course objectives. The books are:

\begin{enumerate}
\def\labelenumi{\arabic{enumi}.}
\tightlist
\item
  Brewer, Cynthia. 2015. \emph{Designing Better Maps: A Guide for GIS users}. Redlands, CA: ESRI Press.

  \begin{itemize}
  \tightlist
  \item
    This book is \emph{required} and can be purchased in the bookstore or online
  \end{itemize}
\item
  Wickham, Hadley and Garrett Grolemund. 2017. \emph{R for Data Science}. O'Reily Media: Sebastopol, CA.

  \begin{itemize}
  \tightlist
  \item
    This book is \emph{optional} and can be purchased in the bookstore, online, or accessed for free \href{http://r4ds.had.co.nz}{as a webbook}. Because it is optional, I recommend using the free version.
  \end{itemize}
\end{enumerate}

I do not require students to buy physical copies of texts. You are free to select a means for accessing these texts that meets your budget and learning style. If eBook editions (e.g.~Kindle, iBooks, \texttt{pdf}, etc.) of texts are available, they are acceptable for this course. All texts should be obtained in the edition noted above.

All readings are listed on the \textbf{Reading List} and should be completed before the course meeting on the week in which they are assigned. Full text versions of most readings not found in the books assigned for the course will be linked to in the Syllabus. For one or two readings, \texttt{.pdf} copies will be made available via GitHub.

\hypertarget{computers-data-and-data-storage}{%
\section{Computers, Data, and Data Storage}\label{computers-data-and-data-storage}}

Since we are meeting virtually, you must have access to a computer that you have Administrator rights for so that you can install the software that is needed for the course. Spatial data files tend to be large, and this course will involve tens of gigabytes worth of data. You'll need at least 20GB free on your computer for this course. \textbf{If you do not have a computer, or do not have reliable home internet, please let Chris know as soon as possible.}

\hypertarget{services}{%
\section{Services}\label{services}}

Over the course of the semester, we'll use two web-based services. Each of these will require you to create an account with a username and password. GitHub will require you to enable \href{https://en.wikipedia.org/wiki/Multi-factor_authentication}{two-factor authentication} as well. I strongly recommend using a \href{https://lifehacker.com/5529133/five-best-password-managers}{password manager}.

\hypertarget{blackboard}{%
\subsection{Blackboard}\label{blackboard}}

SLU is requiring courses to use Blackboard during the pandemic. I'll be sharing lecture recordings here along with links to other resources (under ``Course Materials''), posting grades, and using the message board function for answering questions between classes. You can access Blackboard using your mySLU username and password.

\hypertarget{zoom}{%
\subsection{Zoom}\label{zoom}}

We'll be meeting weekly through Zoom. You must use Zoom with your SLU account (as opposed to a personal account you may have), and can access Zoom via mySLU. Meeting information is available on Blackboard.

\hypertarget{github}{%
\subsection{GitHub}\label{github}}

The majority of course content (sample code, documentation, and assignments) for this course will be made available using \textbf{\href{http://www.github.com}{GitHub}}. GitHub is a website used by programmers, data analysts, and researchers to share computer code and projects. GitHub will also be used for assignment submission and feedback. In addition to providing us with platform for hosting course content, using GitHub will give you experience in some of the techniques that researchers use to conduct both open-source and collaborative research. GitHub is free to use but does have some premium features, which students can access for free through their \href{https://education.github.com/pack/}{Student Developer program}. As I noted above, these premium features \emph{are not required} for this course but are worth knowing about if you decide to continue using GitHub.

\hypertarget{arcgis-online}{%
\subsection{ArcGIS Online}\label{arcgis-online}}

In addition to using the software described below, we'll spend some time using ArcGIS Online. As we approach the fourth module of the course, I will create usernames and temporary passwords for each of you. More details will be provided as we progress through the semester!

\hypertarget{software}{%
\section{Software}\label{software}}

There are two key applications we'll be using this semester in addition to the services listed previously: \href{https://www.rstudio.com}{RStudio} and \href{https://desktop.github.com}{GitHub Desktop}. Both RStudio and GitHub Desktop of these are open-source applications that can be downloaded and used without cost.

\hypertarget{r-and-rstudio}{%
\subsection{\texorpdfstring{\texttt{R} and RStudio}{R and RStudio}}\label{r-and-rstudio}}

The primary tool we will use for data manipulation and analysis is the programming language \texttt{R}. \texttt{R} is open-source, freely available, and highly extensible analysis environment. We'll use \href{https://www.rstudio.com}{RStudio} as the ``front end'' for our analyses. RStudio makes it easier to write \texttt{R} code and to produce well documented analyses. Like the \texttt{R} programming language itself, RStudio is freely available.

\hypertarget{github-desktop}{%
\subsection{GitHub Desktop}\label{github-desktop}}

You will need another free application called \href{https://desktop.github.com}{GitHub Desktop}. This program allows you to easily copy data from GitHub onto your computer. It also makes it easy to upload files like labs and problem sets to GitHub. If you have already used Git via the command line, you can continue to do so without utilizing GitHub Desktop.

\hypertarget{course-policies}{%
\chapter{Course Policies}\label{course-policies}}

My priority is that class periods are productive learning experiences for all students. In order to foster this type of productive environment, I ask students to follow a few general policies and expectations:\footnote{These general expectations were adopted from language originally used by Dr.~Shelley Kimmelberg.}

\begin{enumerate}
\def\labelenumi{\arabic{enumi}.}
\tightlist
\item
  Work each week to contribute to a positive, supportive, welcoming, and compassionate class environment.
\item
  Arrive to class on time and stay for the entire class period.
\item
  Silence \emph{all} electronic devices before entering the classroom.
\item
  Do not engage in side conversations. This is disrespectful to the speaker (whether me or a classmate), and can affect the ability of others in the class to learn.
\item
  Be respectful of your fellow classmates. Do not interrupt when someone is speaking, monopolize the conversation, or belittle the ideas or opinions of others.
\item
  Complete the assigned readings for each class in advance, and come prepared with discussion points and questions.
\end{enumerate}

The following sections contain additional details about specific course policies related to attendance, participation, electronic device use, student support, academic honesty, and Title IX.

\hypertarget{compassionate-coursework}{%
\section{Compassionate Coursework}\label{compassionate-coursework}}

\begin{quote}
Being around people who are different from us makes us more creative, more diligent and harder-working
\end{quote}

\textbf{\href{https://www.scientificamerican.com/article/how-diversity-makes-us-smarter/}{Katherine Phillips, 2014}}

The goal of this course is not just to impart knowledge related to GIS and data science, but to purposefully create an environment where all students feel welcome and supported even as they feel challenged intellectually.\footnote{Much of this approach to ``compassionate coursework'' is adopted from \href{https://twitter.com/aprilwensel}{April Wensel's} work at \href{https://compassionatecoding.com}{Compassionate Coding}.} This is especially important in a STEM course, where stress levels among students can be generally high. For those of you who have not created maps or written computer code before, that could be enough to treat GIS coursework with apprehension. Feeling at least a little anxious about a course like this is understandable and to be expected.

In response to this stress, students sometimes develop ``impostor syndrome'', a feeling that academic gains are not the result of their own abilities and a fear that they will soon be ``found out'' (\href{https://www.physiology.org/doi/10.1152/advan.00085.2017}{Cooper et al.~2018}, \href{http://genderandset.open.ac.uk/index.php/genderandset/article/view/435}{Lindemann et al.~2016}). This is reported with particular frequency by students from social groups traditionally underrepresented in STEM courses (\href{https://onlinelibrary.wiley.com/doi/pdf/10.1002/sce.20307}{Malone and Barabino 2008}, \href{https://doi.org/10.1525/sp.2005.52.4.593}{Ong 2005}, \href{https://doi.org/10.17763/haer.81.2.t022245n7x4752v2}{Ong et al.~2011}). Taking these concerns seriously is imperative not just for reasons of academic retention and its future implications (\href{https://open.nytimes.com/why-having-a-diverse-team-will-make-your-products-better-c73e7518f677}{Akinnawonu 2017}, \href{https://www.tandfonline.com/doi/abs/10.5172/impp.2013.15.2.149}{Diaz-Garcia et al.~2011}, \href{https://www.aauw.org/aauw_check/pdf_download/show_pdf.php?file=why-so-few-research}{Hill et al.~2010}, \href{https://www.tandfonline.com/doi/abs/10.1111/ecge.12016}{Nathan and Lee 2015}), but also because we are called to do so by the \href{https://www.slu.edu/about/catholic-jesuit-identity/mission.php}{University's mission} both in our classrooms and in the wider world.

If you are feeling stressed about the coursework, feel like it is taking what seems like an excessive amount of time, or want to talk about strategies for problem solving, please reach out during class, office hours, or via Discourse. This will be my eleventh semester teaching research methods, and I have plenty of strategies for success in these courses that I am happy to share!

\hypertarget{code-of-conduct}{%
\subsection{Code of Conduct}\label{code-of-conduct}}

While I take a leading role in fostering a welcoming and supportive environment, I need each student's help in making that environment a reality. To that end, you should familiarize themselves with \href{https://www.contributor-covenant.org}{Contributor Covenant's} \href{https://www.contributor-covenant.org/version/1/4/code-of-conduct}{Code of Conduct}, which is increasingly included in open source projects and is included with each lecture repository on \href{https://github.com/slu-soc5650}{GitHub}. The Code of Conduct lays out expectations for how all students should to conduct themselves. I want to emphasize one piece here in the syllabus, which includes concrete examples of things each student \emph{can} and \emph{should} do to help create a compassionate class atmosphere:

\begin{quote}
Examples of behavior that contributes to creating a positive environment include: using welcoming and inclusive language, being respectful of differing viewpoints and experiences, gracefully accepting constructive criticism, focusing on what is best for the {[}class{]}, {[}and{]} showing empathy towards other community members
\end{quote}

The degree to which students are positively engaged with our class along these lines will be reflected in participation grades given at the mid and end points of the semester. If you feel that a colleague's conduct is not in line with creating compassionate coursework experience, you are encouraged to speak to me. I will treat all discussions with discretion and will work with you to make a plan for addressing any concerns you might have.

\hypertarget{harrassment-and-title-ix}{%
\subsection{Harrassment and Title IX}\label{harrassment-and-title-ix}}

While I have every expectation that each member of the Saint Louis University community is capable and able to treat community members with dignity and respect, I fully recognize that there may be instances where students fall short of that expectation. Students should generally be aware that:

\begin{quote}
Saint Louis University prohibits harassment because of sex, race, color, religion, national origin, ancestry, disability, age, sexual orientation, marital status, military status, veteran status, gender expression/identity, genetic information, pregnancy, or any other characteristics protected by law.
\end{quote}

All students should also familiarize themselves with \href{http://www.slu.edu/general-counsel-home/office-of-institutional-equity-and-diversity}{Saint Louis University's polices} on bias, discrimination, harassment, and sexual misconduct. In particular, they should be aware of policies on \href{https://www.slu.edu/general-counsel/institutional-equity-diversity/pdfs/harassment-policy.pdf}{harassment} and \href{https://www.slu.edu/about/safety/sexual-assault-resources.php}{sexual misconduct}:

\begin{quote}
Saint Louis University and its faculty are committed to supporting our students and seeking an environment that is free of bias, discrimination, and harassment. If you have encountered any form of sexual misconduct (e.g.~sexual assault, sexual harassment, stalking, domestic or dating violence), we encourage you to report this to the University. If you speak with a faculty member about an incident of misconduct, that faculty member must notify SLU's Title IX Coordinator, Anna R. Kratky (DuBourg Hall, Room 36; \href{mailto:anna.kratky@slu.edu}{\nolinkurl{anna.kratky@slu.edu}}); 314-977-3886) and share the basic facts of your experience with her. The Title IX coordinator will then be available to assist you in understanding all of your options and in connecting you with all possible resources on and off campus.
\end{quote}

\begin{quote}
If you wish to speak with a confidential source, you may contact the counselors at the University Counseling Center at 314-977-TALK.
\end{quote}

Instances of abusive, harassing, or otherwise unacceptable behavior should be reported either directly to the instructor or to the University Administration. Consistent with the above policies, I will forward all reports of inappropriate conduct to the Title IX Coordinator's office or to the Office of Diversity and Affirmative Action. Please be aware that University policies may require me to forward information about the identity of any students connected to the disclosure.

Please also be aware that communications over various online services, including (but not limited to) Blackboard forums and GitHub, are covered by this policy.

\hypertarget{attendance-and-participation}{%
\section{Attendance and Participation}\label{attendance-and-participation}}

Attendance and participation are important components of this course since we only meet once a week. Students are expected to attend all class sessions; missing even one class can create a significant roadblock for many students. If you cannot attend class or arrive on time because of a personal illness, a family issue, jury duty, an athletic match, or a religious observance, you must contact me \textbf{beforehand} to let me know. I may ask for more information, such as a note from a physician, a travel letter from Athletics, or other documentation for absences.

A penalty will not be applied to your first unexcused absence or late arrival. Any absences or late arrivals beyond the first will result in no credit (for an absences) or only partial credit (for a late arrival) being earned for that day's participation grade.

Making up missed classes are your responsibility. I do post slide decks on the course website, but my slides are intended only to serve as references. I also provide recordings of lectures using Panopto, but these may be incomplete or may not capture key aspects of lectures (such as things that come up during the lab sessions).

The academic literature (see \href{https://link.springer.com/article/10.1007/s10734-018-0275-9}{this recent article} for a nice overview) suggests that the impact of lecture capture tools like Panopto is mixed at best. Students tend to rate their experience with lecture capture far more positively than faculty do. Students who continue to attend class regularly and use lecture recordings as part of an \emph{active} approach to studying and filling in specific areas of their notes may benefit from recordings. Students who use recordings as a replacement for attending class do not benefit from the recordings.

Please also note that lectures and discussions cannot be recorded by any means (e.g.~audio or video recordings, or photographs) without my permission.

\hypertarget{communication}{%
\section{Communication}\label{communication}}

Our course Blackboard message board and email are my preferred methods of communication.

I will the check Blackboard during weekday business hours and will respond to messages if I am able. Likewise, I dedicate time to email responses each workday, meaning that my response time is typically within 24 hours during the workweek. Please use your SLU email account when emailing me.

For both email and Blackboard, if you have not received a response from me after 48 hours (or by end of business on Monday if you messaged me over the weekend), please follow-up to ensure that your message did not get lost in the shuffle.

All messages regarding course updates, assignments, and changes to the class schedule, including cancellations, will be posted on the main Annoucements page on Blackboard. Changes to the class schedule, including cancellations, will also sent to your SLU email account. It is imperative that you check both the forums and your SLU email account regularly.

Please also ensure that all concerns or questions about your standing in the course are directed to me immediately. Inquires from parents, SLU staff members, and others will not be honored.

\hypertarget{electronic-devices}{%
\section{Electronic Devices}\label{electronic-devices}}

During class periods, students are asked to refrain from using electronic devices (including cell phones) for activities not directly related to the course. For this class, I expect students to limit their use of electronic devices to accessing course software, readings, and notes.

There is evidence that using electronic devices during lectures results in decreased retention of course content (\href{https://link.springer.com/article/10.1007/BF02940852}{Hembrooke and Gay 2003}) and lower overall course performance (\href{https://www.sciencedirect.com/science/article/pii/S0360131506001436}{Fried 2008}). Students who are not using a laptop but are in direct view of another student's laptop also have decreased performance in courses (\href{https://www.sciencedirect.com/science/article/pii/S0360131512002254}{Sana et al.~2013}). Conversely, students who take notes the ``old fashioned way'' have better performance on tests compared to students who take notes on laptops (\href{http://journals.sagepub.com/doi/abs/10.1177/0956797614524581}{Mueller and Oppenheimer 2014}).

I therefore ask students to be conscious of how they are using their devices, the ways such use impacts their own learning, and the effect that it may have on others around them. I reserve the right to alter this policy if electronic device use becomes problematic during the semester.

\hypertarget{student-support}{%
\section{Student Support}\label{student-support}}

\hypertarget{basic-needs}{%
\subsection{Basic Needs}\label{basic-needs}}

If you have difficulty affording groceries or accessing sufficient food to eat every day, or lack a safe and stable place to live, you are urged to contact the \href{https://www.slu.edu/student-development/dean-of-students/index.php}{Dean of Students} for support. Likewise if you have concerns about your mental or physical health needs, or lack access to health care services you require, you should contact either the \href{https://www.slu.edu/student-development/dean-of-students/index.php}{Dean of Students}, \href{https://www.slu.edu/life-at-slu/student-health/index.php}{Student Health Services}, or the \href{https://www.slu.edu/life-at-slu/university-counseling/index.php}{University Counseling Center}.\footnote{This language is adopted from text written by \href{https://medium.com/@saragoldrickrab/basic-needs-security-and-the-syllabus-d24cc7afe8c9}{Dr.~Sarah Goldrick-Rab}.}

If you feel comfortable doing so, please discuss any concerns you might have with me. Doing so is particularly important if believe your performance in this course might be affected. I will do my best to work with you to come up with a plan for successfully completing the course and, if need be, work with you to identify on-campus resources. I will treat all discussions with discretion, though please be aware that certain situations, including disclosures of \href{/compassionate-coursework.html}{sexual misconduct} or self harm, must be reported by faculty to the appropriate University office.

\hypertarget{academic-accommodations}{%
\subsection{Academic Accommodations}\label{academic-accommodations}}

If you meet the eligibility requirements for academic accommodations through the \href{https://www.slu.edu/life-at-slu/student-success-center/disability-services/index.php}{Office of Disability Services} (located within the Student Success Center) \emph{and you wish to use them for this class}, you should arrange to discuss your needs with me after the first class. All discussions of this nature are treated confidentially, and I will make every effort to work with you to come up with a plan for successfully completing the course requirements. Please note that I will not provide accommodations to students who are not working with Disability Services, and that I cannot retroactively alter assignments or grades if they have already been completed.

\hypertarget{writing-services}{%
\subsection{Writing Services}\label{writing-services}}

I also encourage you to take advantage of the \href{https://www.slu.edu/life-at-slu/student-success-center/academic-support/university-writing-services/index.php}{University Writing Services (UWS) program}. Getting feedback benefits writers at all skill levels and the quality of your writing will be reflected in assignment grades. The UWS has trained writing consultants who can help you improve the quality of your written work. UWS's consultants are available to address everything from brainstorming and developing ideas to crafting strong sentences and documenting sources.

\hypertarget{student-athletes}{%
\subsection{Student-athletes}\label{student-athletes}}

If you are a student-athlete who is in-season, you should discuss your game schedule with me after the first class and share your travel letter with me as soon as you have a copy. You are reminded that games and tournaments are not excuses for failing to complete assignments, and that NCAA rules prohibit student-athletes from missing classes for practice. Low grades that jeopardize eligibility must be addressed immediately by you, not by a coach or academic coordinator.

\hypertarget{academic-honesty}{%
\section{Academic Honesty}\label{academic-honesty}}

All students should familiarize themselves with \href{http://www.slu.edu/Documents/provost/academic_affairs/Academic\%20Integrity\%20Policy\%20FINAL\%20\%206-26-15.pd}{Saint Louis University's policies} concerning cheating, plagiarism, and other academically dishonest practices:

\begin{quote}
Academic integrity is honest, truthful and responsible conduct in all academic endeavors. The mission of Saint Louis University is ``the pursuit of truth for the greater glory of God and for the service of humanity.'' Accordingly, all acts of falsehood demean and compromise the corporate endeavors of teaching, research, health care, and community service via which SLU embodies its mission. The University strives to prepare students for lives of personal and professional integrity, and therefore regards all breaches of academic integrity as matters of serious concern.
\end{quote}

Any work that is taken from another student, copied from printed material, or copied the internet without proper citation is expressly prohibited. Note that this includes all computer code, narrative text, and documentation written for class assignments - each student is expected to author and de-bug their notebooks and accompanying files.

All relevant assignments should include in-text citations and references formatted using the \href{https://owl.english.purdue.edu/owl/resource/583/1/}{American Sociological Association (ASA)} style guidelines. Any student who is found to have been academically dishonest in their work risks failing both the assignment and this course.

\hypertarget{success-in-this-course}{%
\chapter{Success in this Course}\label{success-in-this-course}}

Students often ask me how to do well in various aspects of the course, and so this section features my \emph{suggestions} for a successful semester. \emph{These observations are provided with no warranty} - following them does not guarantee any particular outcome. You could do everything in here and still do poorly in the course, and conversely you could ignore much of what is in discussed in the links below and still do well. However, \emph{most} of the students who are successful in this course will follow \emph{most} of these ideas consistently.

These tips are an effort to illuminate what sociologists refer to as the ``\href{https://books.google.com/books?hl=en\&lr=\&id=5r-TAgAAQBAJ\&oi=fnd\&pg=PP1\&dq=hidden+curriculum\#v=onepage\&q=hidden\%20curriculum\&f=false}{hidden curriculum}'' of higher education - there are things you need to do to be successful, but they are often unstated or not clearly communicated.

Since I give this advice out in multiple classes, the documents themselves are stored on my \href{https://chris-prener.github.io}{personal website}:

\begin{enumerate}
\def\labelenumi{\arabic{enumi}.}
\tightlist
\item
  \href{https://chris-prener.github.io/resources/students/little-things/}{Doing the Little Things Right}
\item
  \href{https://chris-prener.github.io/resources/students/letters/}{Letters of Recommendation}
\end{enumerate}

If there are other topics you have questions about, please let me know. These documents are a work in progress.

\hypertarget{assignments-and-grading}{%
\chapter{Assignments and Grading}\label{assignments-and-grading}}

This section provides general details on the different types of assignments for this course. It also contains policies for submitting work, receiving feedback, and late work.

\hypertarget{assignments}{%
\section{Assignments}\label{assignments}}

Your grade for this course will consist of a number of different assignments on which points may be earned. Each category of assignment is described below.

\hypertarget{attendance-and-participation-1} of your final
grade.
\end{rmdtip}

Both attendance and participation are critically important aspects of this class. The class participation grade will be based on (a) attendance, (b) level of engagement during lectures and labs, (c) level of engagement on Slack, and (d) the completion of other exercises including ``entry'' and ``exit'' tickets, the student information sheet, a pre-test, and an end of the semester course evaluation.

Each of these elements is assigned a point value and assessed using a scale that awards full, partial, or no credit. Your participation grade will be split, with 50 points (5\% of your final grade) for the first half of the semester (through Lecture-08) and another 50 points (5\%) for the second half. Since the number of points awarded for participation are variable, the total number of points earned for each half will be converted to a 0 to 50 scale.

I provide the final number of points earned for each half of the course. If you would like a more detailed breakdown of your participation grade, please reach out and I will provide one.

\hypertarget{lab-exercises} of your final grade.
\end{rmdtip}

Each course meeting will include time dedicated to practicing the techniques and applying the theories described in the meeting's materials. These exercises will give you an opportunity to practice skills that correspond with the first four course objectives. Instructions for the labs will be posted in the lecture repositories on \href{https://github.com/slu-soc5650}{\textbf{GitHub}} and will be linked to from the lecture pages on the \href{https://slu-soc5650.github.io/}{\textbf{course website}}. The instructions will also detail the deliverables to be submitted to demonstrate completion of each assignment. Replication files are also provided in the lecture repositories on \href{https://github.com/slu-soc5650}{\textbf{GitHub}}.

There will be a total of fourteen lab exercises over the course of the semester, each of which is worth 2\% of your final grade. Lab exercises are graded using the ``check'' grading system. Since replication files are posted, feedback for labs is not generally returned after the first few assignments and I will only respond with the number of points awarded if you do not earn full credit. \emph{It is your responsibility to reach out to discuss questions or concerns you had while doing the lab.}

\hypertarget{final-project} of your final grade.
The waypoints are worth \textbf{28\%} of your final grade, and the final
deliverables are worth \textbf{34\%} of your final grade. Deliverables
vary by section; see below for details.
\end{rmdtip}

The final project corresponds with the fourth learning outcome. It will be organized slightly differently depending on which section you are enrolled in. Specific instructions will be provided in the \href{https://slu-soc5650.github.io/final-project}{\textbf{final project guide}}.

As work progresses, there will be a number of \textbf{waypoints} where students will need to submit updates on their progress. Waypoints serve as the homework assignments for this class - this is your opportunity to show me how your skills are developing \emph{and} make progress on your final project. They are generally due two weeks after the relevant topic is first introduced in class. The seven waypoints are as follows:

\begin{enumerate}
\def\labelenumi{\arabic{enumi}.}
\tightlist
\item
  Meeting 02 (\textbf{February 8th}) - memo submission
\item
  Meeting 06 (\textbf{March 8th}) - initial data cleaning
\item
  Meeting 07 (\textbf{March 15th}) - combining data sources
\item
  Meeting 08 (\textbf{March 22nd}) - projecting data
\item
  Meeting 10 (\textbf{April 5th}) - geoprocessing data
\item
  Meeting 13 (\textbf{April 26th}) - draft story map
\item
  Flex day (\textbf{May 10th}) - draft handout
\end{enumerate}

Deliverables for each waypoint are described in the \href{https://slu-soc5650.github.io/final-project}{\textbf{final project guide}}. All waypoints are graded using the ``check'' grading system. Final materials will be due on \textbf{May 17th} (during Finals Week). Final deliverables differ by course section.

\hypertarget{soc-4650}{%
\subsubsection{SOC 4650}\label{soc-4650}}

If you are enrolled in SOC 4650, you will need to create a story map and a handout that both feature some introductory information and thematic maps of your topic.

\begin{table}

\caption{\label{tab:unnamed-chunk-4}SOC 4650 Final Project Breakdown}
\centering
\begin{tabular}[t]{llll}
\toprule
Assignment & Points & Quantity & Total\\
\midrule
Waypoints & 20 pts & x7 & 140 pts\\
Final Code \& Docs & 70 pts & x1 & 70 pts\\
Final Storymap & 70 pts & x1 & 70 pts\\
Final Handout & 30 pts & x1 & 30 pts\\
\bottomrule
\end{tabular}
\end{table}

\hypertarget{soc-5650}{%
\subsubsection{SOC 5650}\label{soc-5650}}

If you are enrolled in SOC 5650, you will also need to create a 3000-4000 word final paper written in the style of an empirical journal article. The following additional deadlines apply:

\begin{enumerate}
\def\labelenumi{\arabic{enumi}.}
\tightlist
\item
  Meeting 05 (\textbf{March 1st}) - annotated Bibliography with a minimum of fifteen peer reviewed sources
\item
  Meeting 11 (\textbf{April 12th}) - draft paper due (graded as part of your participation grade)
\item
  Meeting 12 (\textbf{April 19th}) - peer reviews of the draft paper and analysis development due as a GitHub issue in partner's final project repository (graded as part of your participation grade)
\item
  The final paper is due \textbf{May 17th} along with the rest of the final project
\end{enumerate}

Grading for SOC 5650 is broken down as follows:

\begin{table}

\caption{\label{tab:unnamed-chunk-5}SOC 5650 Final Project Breakdown}
\centering
\begin{tabular}[t]{llll}
\toprule
Assignment & Points & Quantity & Total\\
\midrule
Waypoints & 20 pts & x7 & 140 pts\\
Annotated Bibliography & 25 pts & x1 & 25 pts\\
Final Code \& Docs & 40 pts & x1 & 40 pts\\
Final Storymap & 40 pts & x1 & 40 pts\\
Final Handout & 25 pts & x1 & 25 pts\\
\addlinespace
Final Paper & 40 pts & x1 & 40 pts\\
\bottomrule
\end{tabular}
\end{table}

\hypertarget{submission-and-late-work}{%
\section{Submission and Late Work}\label{submission-and-late-work}}

\hypertarget{assignment-submission}{%
\subsection{Assignment Submission}\label{assignment-submission}}

Copies of all assignment requested deliverables should be uploaded to your private assignments repository on \href{https://github.com/slu-soc5650}{GitHub} before class on the day that the assignments are due. All assignments will contain details on required deliverables.

The GitHub submission policy is in place because it facilitates clear, easy grading that can be turned around to you quickly. Submitting assignments in ways that deviate from this policy will result in a late grade (see below) being applied in the first instance and a zero grade for each subsequent instance.

\hypertarget{licensing-of-student-work}{%
\subsection{Licensing of Student Work}\label{licensing-of-student-work}}

All assignment repositories are licensed under a \href{https://creativecommons.org/licenses/by-nc-nd/4.0/}{Creative Commons Attribution-NonCommercial-NoDerivatives 4.0 International License}. This license explicitly gives you copyright to all work you create for this course. The license gives Chris permission to copy your work (such as for grading) and to re-use your work later for non-commercial purposes (such as in-class examples) so long as you are given credit for it. However, your work cannot be used for monetary gain (such as in a textbook) and derivative works based on your work are prohibited.

The syllabus agreement at the end of the Student Information Sheet includes an acknowledgment of this licensing arrangement. If you have questions about this, please contact Chris before submitting the form.

\hypertarget{late-work}{%
\subsection{Late Work}\label{late-work}}

Once the class begins, any assignments submitted will be treated as late. Assignments handed in within 24-hours of the beginning of class will have 15\% deducted from the grade. I will deduct 15\% per day for the next two 24-hour periods that assignments are late. After 72 hours, I will not accept late work. If you cannot attend class because of personal illness, a family issue, jury duty, an athletic match, or a religious observance, you must contact me beforehand to discuss alternate submission of work.

\hypertarget{extra-credit}{%
\section{Extra Credit}\label{extra-credit}}

From time to time I may offer extra credit to be applied to your final grade. I will only offer extra credit if it is open to the entire class (typically for something like attending a lecture or event on-campus). If I offer extra credit, I will typically require you to submit a short written summary of the activity within a week of the event to obtain the credit. When offered, extra credit opportunities cannot be made-up or substituted if you are unable to attend the event.

\hypertarget{grading}{%
\section{Grading}\label{grading}}

Grades will be included with assignment feedback, which will be disseminated through Github's \textbf{Issues} tool. At midterms, Lecture-17, and finals, I will upload a summary of all assignment grades to a new \textbf{Issue} on GitHub.

All grades that use a ``check'' system (the lecture preps, labs, and some aspects of the final project) will be calculated using the following approach. A ``check-plus'' represents excellent work and will get full credit. A ``check'' represents satisfactory work and will get 85\% of the points available for that assignment. A ``check-minus'' represents work that needs substantial improvement and will get 75\% of the points available for that assignment.

I use a point system for calculating grades. The following table gives the weighting and final point totals for all assignments for this course:

\begin{table}

\caption{\label{tab:unnamed-chunk-6}SOC 4650 and 5650 Points Breakdown}
\centering
\begin{tabular}[t]{lllll}
\toprule
Assignment & Points & Quantity & Total & Percent\\
\midrule
Participation & 25 pts & x2 & 50 pts & 10\%\\
Labs & 10 pts & x14 & 140 pts & 28\%\\
Final Project & 310 pts & x1 & 310 pts & 62\%\\
\bottomrule
\end{tabular}
\end{table}

All feedback will include grades that represent number of points earned. If you want to know your percentage on a particular assignment, divide the number of points earned by the number of points possible and then multiply it by 100.

Final grades will be calculated by taking the sum of all points earned and dividing it by the total number of points possible (1,000). This will be multiplied by 100 and then converted to a letter grade using the
following table:

\begin{table}
\caption{\label{tab:unnamed-chunk-7}Course Grading Scale}

\centering
\begin{tabular}[t]{lll}
\toprule
GPA & Letter & Percent\\
\midrule
4.0 & A & 93.0\% - 100\%\\
3.7 & A- & 90.0\% - 92.9\%\\
3.3 & B+ & 87.0\% - 89.9\%\\
3.0 & B & 83.0\% - 86.9\%\\
2.7 & B- & 80.0\% - 82.9\%\\
\bottomrule
\end{tabular}
\centering
\begin{tabular}[t]{lll}
\toprule
GPA & Letter & Percent\\
\midrule
2.3 & C+ & 77.0\% - 79.9\%\\
2.0 & C & 73.0\% - 76.9\%\\
1.7 & C- & 70.0\% - 72.9\%\\
1.0 & D & 63.0\% - 69.9\%\\
0.0 & F & < 63.0\%\\
\bottomrule
\end{tabular}
\end{table}

Borderline grades (i.e.~a grade within half a percentage point of the next highest letter grade) \emph{will} be rounded up before final grade submission at the end of the semester. A grade of 89.6\% would therefore be submitted to SLU as an ``A-'' while a grade of 89.4\% would be submitted to SLU as a ``B+''. The final grade report will include both the original letter grade and the rounded letter grade if applicable.

\begin{rmdwarning}
No chances will be given for revisions of poor grades. Incomplete grades
will be given upon request only if you have a ``C'' average and have
completed at least two-thirds of the assignments. You should note that
incomplete grades must be rectified by the specified deadline or they
convert to an ``F''.
\end{rmdwarning}

\hypertarget{part-reading-list}{%
\part{Reading List}\label{part-reading-list}}

\hypertarget{course-schedule}{%
\chapter{Course Schedule}\label{course-schedule}}

The following is a high-level schedule that details the general topic covered by each lecture.

\begin{table}

\caption{\label{tab:unnamed-chunk-1}SOC 4650 and 5650 Course Overview}
\centering
\begin{tabular}[t]{llll}
\toprule
Module & Meeting & Date & Title\\
\midrule
1 &  &  & **Course Introduction**\\
1 & 01 & February 1st & Course Introduction\\
1 & 02 & February 8th & Map Production Basics\\
1 & 03 & February 15th & Cartography 101\\
2 &  &  & **Data Cleaning**\\
\addlinespace
2 & 04 & February 22nd & Data Cleaning Basics\\
2 & 05 & March 1st & Combining Data Sources\\
3 &  &  & **Geoprocessing**\\
3 & 06 & March 8th & Working with Map Projections\\
3 & 07 & March 15th & Intersect, Select, and Aggregate\\
\addlinespace
3 & 08 & March 22nd & Union, Dissolve, and Merge\\
4 &  &  & **Map Products with ArcGIS**\\
4 & 09 & March 29th & ArcGIS Online\\
4 & 10 & April 5th & Story Maps, Part 1\\
4 & 11 & April 12th & Story Maps, Part 2\\
\addlinespace
5 &  &  & **Map Products with R**\\
5 & 12 & April 19th & Web Mapping with leaflet\\
5 & 13 & April 26th & Static Map Layouts with tmap\\
5 & 14 & May 3rd & Handouts with RMarkdown\\
\bottomrule
\end{tabular}
\end{table}

\hypertarget{scheduling-notes}{%
\subsection{Scheduling Notes}\label{scheduling-notes}}

The course schedule may change as it depends on the progress of the class and the challenges we are confronted by this semester. The web version of this document will be updated to reflect any alterations, but the .pdf version will remain unaltered.

No class activities are scheduled for \textbf{May 10th}. This is a ``flex day,'' which I have left without a scheduled plan to accommodate changes due to COVID-19. If we have to reschedule class days, this date will be used before we begin to remove content from the course plan. I will provide updates on my plans for both of these days as the semester progresses.

\hypertarget{meeting-schedule}{%
\chapter{Meeting Schedule}\label{meeting-schedule}}

Select a module from the menu to see details about topics, readings, and assignments. Additional notes and links to course materials are available through Blackboard, which has dedicated folders for each module and Meeting.

\begin{rmdwarning}
I am in the midst of making significant revisions to this class from
prior years because of the modifications teaching it remotely entails.
The remainder of the reading list will be available during the week of
February 1st.
\end{rmdwarning}

\hypertarget{module-1---course-introduction}{%
\section*{Module 1 - Course Introduction}\label{module-1---course-introduction}}
\addcontentsline{toc}{section}{Module 1 - Course Introduction}

\hypertarget{meeting-01---monday-february-1---course-introduction}{%
\subsection*{Meeting 01 - Monday, February 1 - Course Introduction}\label{meeting-01---monday-february-1---course-introduction}}
\addcontentsline{toc}{subsection}{Meeting 01 - Monday, February 1 - Course Introduction}

\begin{itemize}
\tightlist
\item
  \textbf{Before Class:}

  \begin{itemize}
  \tightlist
  \item
    Complete the Course Onboarding tasks (see Blackboard - Module 1 \textgreater{} Course Onboarding)
  \item
    Complete the Meeting 01 preparatory readings, videos, and entry ticket

    \begin{itemize}
    \tightlist
    \item
      Coral, Lilian. 2016. ``City of Los Angeles GeoHub.'' Presented at the ESRI User Conference, San Diego, CA. (Link)
    \item
      Parker, Hilary. 2017. ``Opinionated analysis development.'' Presented at rstudio::conf, Orlando, FL. (Link)
    \item
      Healy, Kieran. 2018. ``Introduction.'' In \emph{The plain person's guide to plain text}. (Link)
    \item
      Entry Ticket 01 (Link)
    \end{itemize}
  \end{itemize}
\item
  \textbf{After Class:}

  \begin{itemize}
  \tightlist
  \item
    Complete Lab 01 - A Simple Interactive Map
  \item
    Read through the Final Project instructions and draft your initial memo
  \end{itemize}
\item
  \textbf{Optional Additional Readings:}

  \begin{itemize}
  \tightlist
  \item
    Goodchild, Michael. 2010. ``Twenty years of progress: GIScience in 2010.'' \emph{Journal of Spatial Information Science} 1(1):3-20. (Link)
  \item
    Logan, John. 2010. ``Making a place for space: Spatial thinking in social science.'' \emph{Annual Review of Sociology} 38:507-524. (Link)
  \item
    Parker Hilary. 2017. ``Opinionated analysis development.'' \emph{PeerJ Preprints} 5:e3210v1. (Link)
  \item
    Thieme, Nick. 2018. ``R generation.'' \emph{Significance} 15(4):14-19. (Link)
  \end{itemize}
\end{itemize}

\begin{center}\rule{0.5\linewidth}{0.5pt}\end{center}

\hypertarget{meeting-02---monday-february-8---map-production-basics}{%
\subsection{Meeting 02 - Monday, February 8 - Map Production Basics}\label{meeting-02---monday-february-8---map-production-basics}}

\begin{itemize}
\tightlist
\item
  \textbf{Before Class:}

  \begin{itemize}
  \tightlist
  \item
    Complete Lab-01 from last week
  \item
    Complete the Meeting 02 preparatory videos and entry ticket (via Blackboard)

    \begin{itemize}
    \tightlist
    \item
      Prep Video A - Analysis Development
    \item
      Prep Video B - Cartography Basics
    \item
      Submit final project memo (\href{https://slu-soc5650.github.io/final-project/topic-data-selection.html}{Link})
    \item
      Entry Ticket 02
    \end{itemize}
  \end{itemize}
\item
  \textbf{After Class:}

  \begin{itemize}
  \tightlist
  \item
    Complete Lab-02
  \item
    Follow-up with Chris in your Final Project repo about any outstanding memo questions
  \end{itemize}
\item
  \textbf{Optional Additional Readings:}

  \begin{itemize}
  \tightlist
  \item
    Wickham, Hadley. ``Workflow: Projects.'' \emph{R for Data Science}. (\href{https://r4ds.had.co.nz/workflow-projects.html}{Link})
  \item
    Wilson, Greg, Jennifer Bryan, Karen Cranston, Justin Kitzes, Lex Nederbragt, and Tracy Teal. 2017. ``Good enough practices in scientific computing.'' \emph{PLoS Computational Biology} 13(6):e1005510. (\href{https://journals.plos.org/ploscompbiol/article?id=10.1371/journal.pcbi.1005510}{Link})
  \end{itemize}
\end{itemize}

\begin{center}\rule{0.5\linewidth}{0.5pt}\end{center}

\hypertarget{meeting-03---monday-february-15---cartography-101}{%
\subsection{Meeting 03 - Monday, February 15 - Cartography 101}\label{meeting-03---monday-february-15---cartography-101}}

\begin{itemize}
\tightlist
\item
  \textbf{Before Class:}

  \begin{itemize}
  \tightlist
  \item
    Complete Lab-02 from last week
  \item
    Complete the Meeting 02 preparatory videos, readings, and entry ticket (via Blackboard)

    \begin{itemize}
    \tightlist
    \item
      Prep Video A - Visual Contrast
    \item
      Prep Video B - Working with Color (1)
    \item
      Prep Video C - Working with Color (2)
    \item
      Brewer, \emph{Designing Better Maps}, Chapters 7 and 8
    \item
      Entry Ticket 03
    \end{itemize}
  \end{itemize}
\item
  \textbf{After Class:}
\item
  Complete Lab-03
\end{itemize}

  \bibliography{book.bib,packages.bib}

\end{document}
